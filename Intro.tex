In areas of applied statistics, such as econometrics, understanding and quantifying casual relationships is crucial for determining the effect of one event on another. Generally, causal models seek to identify the effect of a treatment or independent variable on an outcome or dependent variable while controlling for confounding factors that could influence this relation. There are a variety of methodologies used to establish causalities and this paper serves to explore the matrix theory behind two of them, Two Stage Least Squares and Double Machine Learning. The role matricies play in these causal models are within their systems of linear equations and how they facilitate the computation of estimators that quantify the causal effects. 
\subsubsection*{History}
In the 40's \& 50's, the methodology for Two-Stage Least Squares (2SLS) was developed as a solution to the problem of endogeneity in linear regression models, which occur when an independent variable is correlated with the error term. This leads to biased and inconsistent OLS estimates and is due to various factors like ommited variable bias, measurement error, or simultaneous causality. 
With the advent of machine learning models and their increased usage, the necessity for establishing causality eventually led to developments such as the methodology called Double Machine Learning (DML) in the 2017 paper "Double/Debiased Machine Learning for Treatment and Causal Parameters\cite{chernozhukov2017doubledebiasedmachinelearningtreatment},
