Overall we have looked at the contextual application for Two Stage Least Squares and Instrumental Variables and the matrix theory involved, further discussing the use of SVD in calculating the psuedo-inverse for it and how that transitions to L2 regularization. From there we establish the fundamental theory of Double Machine Learning and provide an example of DML with L2 regularization. By leveraging the mathematical properties of matrices, these methodologies provide robust solutions to complex problems of endogeneity, multi-collinearity and high-dimensionality in statistical modeling. The analysis demonstrates how matrices facilitate the computation of consistent and unbiased estimators, ultimately contributing to more accurate and reliable causal inferences. Whereas 2SLS and IVs are a fundamental tool used in applied research for Economics, Finance, Accounting, etc. outside of Economics, I have yet to see double machine learning firmly established as a causal methodology; however, I hope to see this methodology used more in all sorts of applied literature in the future.